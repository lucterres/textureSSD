\documentclass{ieeeaccess}
\usepackage{amsmath,amssymb,amsfonts}
\usepackage{graphicx}
\usepackage{textcomp}
\usepackage[utf8]{inputenc}

% BibTeX logo definition according to IEEE Access standard
\def\BibTeX{{\rm B\kern-.05em{\sc i\kern-.025em b}\kern-.08em
    T\kern-.1667em\lower.7ex\hbox{E}\kern-.125emX}}

\usepackage[T1]{fontenc}
\usepackage{lmodern}
\usepackage{mathptmx}

\usepackage{booktabs}
\usepackage{siunitx}

\usepackage{comment}

\usepackage{bm}
\usepackage{algorithm}
\usepackage{algpseudocode}
% Hyperref para criar marcadores (bookmarks) no PDF
\usepackage[
    bookmarks=true,
    bookmarksopen=true,
    bookmarksnumbered=true,
    colorlinks=true,
    linkcolor=blue,
    citecolor=blue,
    urlcolor=blue,
    pdfstartview=FitH,
    pdftitle={Context-oriented Synthesis of Salt Domes in Labeled Seismic Images},
    pdfauthor={Luciano D. Terres, Jacob Scharcanski}
]{hyperref}

%Your document starts from here ___________________________________________________
\begin{document}
\history{Date of publication 2025 12, 0000, date of current version xxxx 00, 0000.}
\doi{10.1109/ACCESS.2024.0429000}

\title{Context-oriented Synthesis of Salt Domes in Labeled Seismic Images}
\author{\uppercase{Luciano D. Terres}\authorrefmark{1} and
\uppercase{Jacob Scharcanski}\authorrefmark{1}, \IEEEmembership{Senior Member, IEEE} }

\address[1]{Institute of Informatics,
Federal University of Rio Grande do Sul, Porto Alegre, RS, Brasil, 91501-970 (e-mail: ldterres@inf.ufrgs.br; jacobs@inf.ufrgs.br)}

\markboth
{Author \headeretal: Preparation of Papers for IEEE TRANSACTIONS and JOURNALS}
{Author \headeretal: Preparation of Papers for IEEE TRANSACTIONS and JOURNALS}

\corresp{Corresponding author: Jacob Scharcanski (e-mail: jacobs@inf.ufrgs.br)}


\begin{abstract}
Advanced remote sensing and seismic image analysis and interpretation methods based on training robust machine learning models, which rely on large annotated datasets. However, obtaining such large annotated datasets is challenging in applications like offshore oil exploration, where expert annotation is costly and time-consuming. This work proposes a context-oriented data augmentation scheme for generating synthetic seismic images of salt dome structures with pixel-level mask labels. The proposed approach uses Variational Autoencoders (VAE) to learn and generate realistic salt body geometries, that are combined with a context-aware non-parametric texture synthesis method, to create seismic textures across three distinct zones: salt bodies, conventional rocks, and their boundaries. Using the TGS Salt Identification Challenge dataset as training data,  several generated synthetic samples were evaluated by three geoscientists and compared with a state-of-the-art seismic image synthesis method. The qualitative evaluation shows that the synthetic images are virtually indistinguishable from the real ones, with the experts F1-scores differing by less than 2\%. Furthermore, quantitative comparisons demonstrate substantial improvements over the baseline, achieving significantly lower pixel-wise error, higher structural similarity, and better texture fidelity. These results suggest the effectiveness of the proposed context-oriented approach for generating high-quality synthetic seismic training data, which is crucial for seismic image analysis and interpretation.
\end{abstract}

\begin{keywords}
Data augmentation, deep learning, image segmentation, seismic images, seismic interpretation, texture synthesis, and variational autoencoders.
\end{keywords}

\titlepgskip=-21pt

\maketitle

\section{Introduction}
\label{sec:introduction}
\PARstart{A} relevant task in seismic imaging analysis and interpretation is to distinguish accurately saline bodies from other sediments. However, seismic reflections in salt rocks pose a challenge to seismic imaging researchers due to their distinct acoustic characteristics and complex geometry. The boundaries of saline bodies can be identified by trained experts~\cite{Zeng2019}, but this process involves massive amounts of seismic data and is labor intensive, making it a potential candidate for being handled by smart image segmentation methods. Nevertheless, public databases of annotated seismic images are scarce, and such data is needed for training smart seismic image segmentation methods based on machine learing schemes, such as RESNET~\cite{He2016} and UNET~\cite{Ronneberger2015}, which are specialized in image segmentation, analysis and recognition. 

Therefore, researchers have recently directed their efforts towards data augmentation through seismic image synthesis~\cite{Henriques2021}. Seismic images present unique challenges due to their non-stationary nature, where texture patterns vary significantly across different geological structures. Salt bodies, in particular, exhibit distinct acoustic properties that generate characteristic boundary patterns in seismic data. Understanding these characteristics is essential for developing effective synthesis methods that are able to create realistic seismic images.

In this work, we propose a mechanism for synthesizing annotated seismic image samples containing salt domes. We use a combination of Variational Autoencoder (VAE) and a context-oriented texture synthesis method to generate new seismic samples containing salt domes. The VAE scheme comprises an encoder that maps the multidimensional input into a low-dimensional latent space, and a decoder that reconstructs the output from these latent features. We wish to minimize the difference between the input and the reconstructed seismic image samples~\cite{Li2020}. On the other hand, visual texture synthesis learns to generate new seismic samples from a particular seismic image by inferring its generation process directly from the seismic image examples provided. The generated textures should be indistinguishable from real data from the human experts point of view. If the experts cannot distinguish between an original texture and a synthesized one, the synthesis process is considered successful~\cite{Gatys2015}.

The main research hypothesis of this work is that the combination of VAEs for generating geometric masks with context-oriented texture synthesis can generate synthetic seismic images of salt domes that are virtually indistinguishable from real seismic images by geoscience experts. This hypothesis addresses a critical scientific problem in the field of seismic imaging, analysis and interpretation. This scientific problem arises from the fundamental requirements of modern deep learning models for salt body segmentation, analysis and interpretation. These models demand large volumes of annotated data to achieve robust performance and generalization. However, seismic image annotation is an expensive and time-consuming process that requires specialized expertise and trained geoscientists. The limited availability of annotated datasets significantly restricts the model generalization capabilities, which is particularly concerning given that offshore oil exploration critically depends on the precise identification of salt structures for successful reservoir characterization and drilling operations. This work innovates by combining VAEs to generate salt and rock shapes with a context-oriented texture synthesis process, which handles non-stationary textures of salt bodies, conventional rocks, and their boundaries. 

The remainder of this paper is organized as follows. Section~\ref{sec:related} reviews the related work in seismic image synthesis and generative models. Section~\ref{sec:methodology} presents the proposed methodology, detailing the VAE architecture for context generation, the context-oriented texture synthesis approach, and the complete seismic image  synthesis pipeline. Section~\ref{sec:results} discusses the experimental setup, dataset description, evaluation measures, and presents both qualitative and quantitative results, including a comprehensive comparison with the state-of-the-art method available. Section~\ref{sec:ablation} presents an ablation study analyzing the contribution of each component of the proposed methodology. Finally, Section~\ref{sec:conclusion} concludes the paper with a discussion of practical applications and suggests directions for future work.


\section{Related Work}
\label{sec:related}

\PARstart{O}{ver} the past decades, a variety of algorithms have been proposed to address the challenges posed by seismic image synthesis. In the early days, before the rise of neural networks, researchers relied on traditional methods like model-based approaches and basic texture synthesis techniques. These methods worked reasonably well for simple tasks,  but often struggled with complex geological features and the non-uniform textures commonly found in seismic data. The primary limitation of these early techniques was their inability to handle non-stationary textures, which are textures that change across the different regions of a seismic image~\cite{Zhou2018}.

In recent years, the introduction of deep learning techniques in seismic image synthesis has advanced significantly  the field of seismic image synthesis~\cite{Zeng2019}. The introduction of Convolutional Neural Networks (CNNs) was a major turning point, since it brought a new level of sophistication to seismic image synthesis~\cite{Powers2011}. CNNs excel at extracting detailed features from images, such as horizontal and vertical lines, and can even synthesize entire objects within a seismic image. The CNNs capabilities allowed for more accurate and realistic synthesized seismic images,  making it easier to train methods for the analysis and interpretation of subsurface structures~\cite{Powers2011}. 

However, the success of CNNs came with a significant challenge: learning schemes for synthesizing such seismic image structures require large, annotated datasets for training. In seismic imaging, such datasets tend to be unavailable, or difficult to produce. Therefore, researchers have explored various alternatives, including data augmentation with synthetic data and transfer learning. Despite these efforts, the issue of limited training data remains a hurdle in the widespread application of CNNs to seismic image synthesis~\cite{Wang2021}.

More recently, R. S. Ferreira \emph{et al.}~\cite{Ferreira2020} proposed a scheme for generating synthetic seismic images. In addition to using expert evaluation, they also evaluated their results using measures that estimate the differences between synthetic and real images. Their work addresses the generation of synthetic seismic images of various geological fields, including geological faults and salt domes. Since the work of R. S. Ferreira \emph{et al.}~\cite{Ferreira2020} approaches the same problem and is a clear representative of the state-of-the-art in the field, we compare their results with the results obtained by the proposed scheme using the same measures and methodologies, evaluating the realism of synthetic seismic images when contrasted with real seismic images.

Similarly, Henriques \emph{et al.}~\cite{Henriques2021} proposed a very similar data augmentation method based on training two generative models to augment the number of samples in a seismic image dataset for the semantic segmentation of salt bodies. However, they perform an indirect evaluation of results, based on the improvement in the performance of segmentation methods based on convolutional neural networks, which could be difficult to use as a direct comparison of results with our proposed method and on a different database.

Choi \emph{et al.}~\cite{Choi2025} address the scarcity of labeled seismic data for training deep learning models by employing advanced generative techniques. They propose a method for generating synthetic seismic images using diffusion models, a class of generative models proficient in various image synthesis tasks. While their primary focus is on generating synthetic seismic data for fault detection, their conditional diffusion model incorporates geological context, ensuring the synthetic images are not only realistic, but also are geologically plausible. Their study demonstrates that diffusion models can effectively capture the complex patterns and textures in seismic imaging data, yielding high-quality synthetic images. Their results confirm that synthetic seismic images enhance the training of fault detection algorithms, leading to an improved performance in identifying subsurface structures.

In summary, while there has been significant progress in the field of seismic image synthesis, some challenging issues  remain open in terms of data availability, evaluation methodologies, and the need for more robust generative models. Our work addresses these challenges by leveraging VAEs and advanced texture synthesis techniques to produce high-quality synthetic seismic images.



\section{Proposed Methodology}
\label{sec:methodology}

%\subsection{Overview of the Approach}

\PARstart{T}{he} proposed context-Oriented Seismic Image Synthesis methodology combines a deep learning model based on a VAE, with a texture synthesis algorithm. It synthesizes seismic image samples for data augmentation within a specific context mask. The proposed approach consists of two main stages: (1) context generation using a VAE that generates salt body masks corresponding to a given salt geometry, and (2) texture synthesis using a non-parametric algorithm that creates realistic seismic images based on the generated masks. The proposed method focuses on context zones and on the characteristics of seismic images with salt domes, ensuring that the synthetic samples preserve the structural and textural properties of real seismic imaging data. Figure~\ref{fig:synthesis_pipeline} illustrates the complete synthesis pipeline, showing how the VAE-generated masks are combined with context-oriented texture synthesis to produce realistic synthetic seismic images.

\begin{figure}[htbp]
    \centering
    \includegraphics[width=1\linewidth]{images/terre1.png}
    \caption{Complete pipeline of the proposed context-oriented seismic image synthesis methodology, showing the integration of VAE-based mask generation and the context-aware texture synthesis method.}
    \label{fig:synthesis_pipeline}
\end{figure}


\subsection{Context Generation Using a Variational Autoencoder}

The proposed data augmentation scheme detects and generates new rock geometries with a VAE, as illustrated in Fig.~\ref{fig:vae1}.  
\begin{figure}[htbp]
    \centering
    \includegraphics[width=1\linewidth]{images/terre2.png}
    \caption{Samples of generated structural salt masks. The Variational Autoencoder learns the features distribution and produces new structural masks.}
    \label{fig:vae1}
\end{figure}

VAEs are deep generative models that learn to encode seismic imaging data into a lower-dimensional latent space, and then decode it back to the original space~\cite{Kingma2014}. VAEs are particularly suitable for generating new seismic imaging samples that follow the same distribution as the training data, making them ideal for data augmentation tasks in seismic imaging.

Fig.~\ref{fig:vae1} illustrates a dataset of seismic images used to generate new salt dome masks. The natural intersection between two different rocks are their boundaries, and in the case of salt rocks, this boundary usually is apparent, as Fig.~\ref{fig:edge1} shows. During the VAE training process, we use only salt masks with a considerable amount of salt, varying from 10 to 90 percent of the total area.
\begin{figure}[htbp]
    \centering
    \includegraphics[width=1\linewidth]{images/terre3.png}
    \caption{Sample of the boundaries identification between salt and other sediments represented in a dilated edge segment.}
    \label{fig:edge1}
\end{figure}

We use the generated masks as contexts for the texture synthesis process. Once we locate the boundaries of a salt body, we assume that its interior region is constituted by salt. The VAE model used in this work employs a simple feedforward network for the encoder and decoder. The encoder contains four stacked dense layers, and the decoder contains five stacked dense layers. During inference, we sample the latent variables $z$ from the prior distribution $\pi(z)$, and then decode $z$ with the VAE's decoder to generate the salt masks with the inferred distribution.

The main task of VAEs is to model the underlying probability distribution of a data set $\bm{x} = \{x^{(1)}, x^{(2)}, \ldots, x^{(N)}\}$, where each $x^{(i)}$ represents an individual sample. The model introduces a latent variable $\bm{z}$ (a lower-dimensional representation), and parameters $\theta$ to maximize the marginal likelihood $p_\theta(x)$, where $x$ is the observed data:

\begin{equation}
p_\theta(x) = \int p_\theta(x | z) p(z) \, dz,
\end{equation}
where $p_\theta(x | z)$ is the likelihood of $x$ given $z$, and $p(z)$ is the prior distribution over the latent variables.


The key VAE components are the following:

\paragraph{Encoder (Recognition Model)}
Maps an input seismic image sample $x$ to a latent variable $z$, and approximates the posterior distribution $q_\phi(z | x)$ (parameterized by $\phi$) assuming it to be Gaussian:

\begin{equation}
q_\phi(z | x) = \mathcal{N}(z; \mu_\phi(x), \Sigma_\phi(x)),
\end{equation}
where $\mu_\phi(x)$ and $\Sigma_\phi(x)$ are the mean and the covariance of the posterior distribution $q_\phi(z | x)$, which are outputs of the encoder network.

\paragraph{Decoder (Generative Model)}
Reconstructs the data from the latent variable $z$, modeling the likelihood $p_\theta(x | z)$ (parameterized by $\theta$) assuming it as Gaussian:

\begin{equation}
p_\theta(x | z) = \mathcal{N}(x; \mu_\theta(z), \Sigma_\theta(z)),
\end{equation}
where $\mu_\theta(z)$ and $\Sigma_\theta(z)$ are the mean and covariance of the likelihood distribution, which are outputs of the decoder network.

\paragraph{Latent Variables Priors $p(z)$}
Usually assumed as a standard normal distribution:

\begin{equation}
p(z) = \mathcal{N}(z; 0, \mathrm{I}),
\end{equation}
where $\mathrm{I}$ is the identity matrix.

\paragraph{Variational Inference} 
Since the exact posterior $p(z | x)$ usually is intractable, VAEs use variational inference to approximate it by optimizing the evidence lower bound (ELBO), where $\text{KL}(\cdot \parallel \cdot)$ denotes the Kullback-Leibler divergence:

\begin{equation}
\log p_\theta(x) \geq \mathbb{E}_{q_\phi(z | x)} \left[ \log p_\theta(x | z) \right] - \text{KL}(q_\phi(z | x) \parallel p(z)),
\end{equation}
and the ELBO has two important components, namely, the reconstruction loss and the Kullback-Leibler (KL) divergence loss, that are detailed next. The reconstruction loss is the expected log-likelihood of the seismic imaging data, which encourages an accurate seismic image sample reconstruction:

\begin{equation}
\mathbb{E}_{q_\phi(z | x)} \left[ \log p_\theta(x | z) \right],
\end{equation}
and the KL divergence loss is the Kullback-Leibler divergence between the approximate posterior and the prior, which acts as a regularizer. For Gaussian distributions, this can be computed in closed form as:

\begin{equation}
\begin{aligned}
\text{KL}(q_\phi(z | x) \parallel p(z))
=  \\
\frac{1}{2} \sum_{i=1}^{d} \left( \mu_\phi(x)_i^2 + \Sigma_\phi(x)_{ii} - \log \Sigma_\phi(x)_{ii} - 1 \right),
\end{aligned}
\end{equation}
where $d$ is the dimensionality of the latent space, and $\mu_\phi(x)_i$ and $\Sigma_\phi(x)_{ii}$ denote the $i$-th component of the mean vector and the $i$-th diagonal element of the covariance matrix, respectively.

To enable gradient-based optimization, we use the re-parameterization trick. Instead of sampling $z \sim q_\phi(z | x)$ directly, we express $z$ as:

\begin{equation}
z = \mu_\phi(x) + \Sigma_\phi(x)^{1/2} \odot \epsilon, \quad \epsilon \sim \mathcal{N}(0, \mathrm{I}),
\end{equation}
where $\epsilon$ is an auxiliary random variable sampled from a standard normal distribution, and $\odot$ denotes element-wise multiplication. This formulation allows gradients to be propagated through $\mu_\phi(x)$ and $\Sigma_\phi(x)$.

Finally, we train VAEs by maximizing the ELBO with respect to the parameters $\theta$ and $\phi$ using stochastic gradient descent. The optimization process tends to balance the reconstruction accuracy with the latent space regularization.


\subsubsection{Motivation for Using Variational Autoencoders (VAEs) in Seismic Image Synthesis}

The primary reasons for using VAEs are a realistic generation of synthetic samples, an increase in dataset diversity, a regularization and control of variability, and an improvement in the generalization of the segmentation model. When we consider specific reasons related to the nature of seismic data and the purpose of sample generation for data augmentation, VAEs enable the generation of synthetic images that preserve the structural characteristics of real seismic data, thereby enhancing the training and performance of segmentation models~\cite{Henriques2021,Wang2021,Anjom2024}.

While VAEs offer several advantages for seismic data synthesis, it is important to consider alternative generative models that have also been explored in the literature. In the following sections, we briefly discuss Generative Adversarial Networks (GANs) and Diffusion Models, highlighting their main limitations in the context of seismic image generation and data augmentation.

\subsubsection{Limitations of Generative Adversarial Networks (GANs) for Seismic Image Synthesis}

\begin{enumerate}
    \item \textbf{Control over Data Variability}:
    \begin{itemize}
        \item It is essential that the generated images preserve the structural patterns of salt bodies. VAEs ensure that synthetic images fall within the expected distribution of real seismic images. GANs, on the other hand, are more prone to generating images with high variability and may occasionally create unrealistic or inconsistent samples~\cite{Zhou2018}.
    \end{itemize}
    \item \textbf{Regularization and Smoothness in Image Generation}:
    \begin{itemize}
        \item VAEs impose a probabilistic structure on the latent space, resulting in smoother transitions between different samples and generating images that maintain coherent structural characteristics. GANs, in turn, can suffer from mode collapse~\cite{Zhou2018}, where they generate only a limited subset of possible data variations, reducing the diversity of the augmented dataset.
    \end{itemize}
    \item \textbf{Training Stability}:
    \begin{itemize}
        \item GANs require a balance between the generator and discriminator, which can make training unstable and more challenging to converge to a distribution that well represents the seismic data~\cite{Zhou2018}.
    \end{itemize}
\end{enumerate}

Although GANs are known for generating high-quality images, we chose VAEs in this work due to their more refined control over data variability, training stability, and preservation of the latent seismic data distribution.

\subsubsection{Limitations of Diffusion Models for Seismic Image Synthesis}
Diffusion models present practical and conceptual disadvantages as compared to VAEs~\cite{Zhou2024}:
\begin{enumerate}
    \item \textbf{Higher Computational Cost}:
    \begin{itemize}
        \item Diffusion models require many sampling steps to generate images, typically hundreds or thousands of denoising steps~\cite{Zhou2024}. This makes them more computationally expensive than VAEs, which generate samples in a single pass through the model.
        \item In seismic applications, where there is a need to generate a large number of samples for data augmentation, the efficiency of VAEs can be a significant advantage. 
    \end{itemize}
    \item \textbf{Need for Large Volume of Training Data}:
    \begin{itemize}
        \item Diffusion models often require large training datasets to capture high-quality distributions~\cite{Zhou2024}. Since labeled seismic images can be limited, VAEs may be more effective in learning useful representations even with a smaller number of samples.
    \end{itemize}
\end{enumerate}

Therefore, targeting efficiency and ease of training, VAEs have simpler and more stable training, as they do not rely on an iterative refinement process like diffusion models. This can be a decisive factor when there are time and computational resource constraints.


\subsection{Context-oriented Texture Synthesis}

Texture synthesis is a technique used in computer graphics and image processing to generate new textures that are visually similar to a given sample texture. Non-parametric texture synthesis methods~\cite{Efros1999} create new images by sampling and recombining patches from sample textures provided as examples, ensuring that the statistical properties of the generated texture closely match those of the texture samples provided as the original  examples.

The typical texture synthesis algorithms use an image database and replicate the images in a picture plane according to some arrangement rules. Also, texture synthesis can be enhanced by working at a finer scale, from the pixel level to a small region level. At each iteration of the algorithm, a small portion of the original texture, a patch, is copied and placed within the newly synthesized texture. Non-parametric sampling~\cite{Efros1999} further improved texture synthesis by creating a new image that shares the same statistical properties as a given sample texture, while avoiding explicit models of the texture by sampling directly from the example texture images.


\subsubsection{Non-parametric Seismic Texture Synthesis Algorithm}

In order to detail the proposed seismic texture synthesis method, the following elements must be defined :

\textbf{1. Neighborhood:} For each pixel $p$ in the image to be synthesized, a region around $p$ already filled with known texture pixel values is defined (namely, a neighborhood $N(p)$);

\textbf{2. Similarity Criterion:} To find corresponding patches in the example texture image, we compute the similarity between $N(p)$ and the neighborhoods $N(q)$ for all pixels $q$ in the example texture image. We define as a similarity measure the Euclidean distance $D$ between the pixels of the neighborhoods:

\begin{equation}
D(N(p), N(q)) = \sum_{i \in N} \left( I_p(i) - I_q(i) \right)^2,
\end{equation}
where $I_p(i)$ and $I_q(i)$ are the pixel intensity values at location $i$ in the neighborhoods of pixels $p$ and $q$, respectively, and $N$ denotes the set of pixel locations in the neighborhood $N(p)$ of a pixel $p$;

\textbf{3. Random Sampling:} We choose a pixel $q$ randomly from the example texture image with probability proportional to its similarity (or inverse of $D(N(p), N(q))$) to $p$. In other words, pixels with smaller distance $D(N(p), N(q))$ are more likely to be chosen. We can model the probability $P(q)$ of selecting pixel $q$ as:

\begin{equation}
P(q) = \frac{\exp(-D(N(p), N(q)) / \sigma^2)}{\sum_{q'} \exp(-D(N(p), N(q')) / \sigma^2)},
\end{equation}
where $\sigma$ is a temperature parameter that controls the influence of distances in the pixel selection probability, and $q'$ denotes all candidate pixels in the example texture image;

\textbf{4. Image Filling:} The process begins with a small initial patch in the new image and gradually expands it, pixel by pixel, using the steps described above to fill the entire texture image that is being synthesized. The choice of the next pixel to be synthesized follows a raster or spiral filling path. The method produces seismic textures that are visually similar to the original input texture. 

\subsubsection{Context-based Seismic Image Synthesis}

We constrain the process of seismic image synthesis to the context masks generated by the VAE model. Various patches are used as inputs, corresponding to different parts of the sample seismic image. The synthesized sample has three main components, namely, salt zone, boundary zone, and conventional rock zone. Initially, we create the boundary, or frontier, zones between the two rocks, which we also call edge zone. We create the boundary zone by using edge detection on the original seismic image to locate the interfaces between the two types of rocks, and then make it thicker, creating an edge strip. As Fig.~\ref{fig:edge1} shows, this strip is an area of high seismic contrast appearing as light and dark bands. Also, various parallel lines make up this boundary, and Fig.~\ref{fig:line1} shows their angles. We use a dataset composed of edge pieces and their corresponding angles as input for the boundary synthesis process. Then, for each edge segment in the mask of the seismic texture image to be synthesized, we synthesize its texture based on the patch with the most similar angle (see Fig.~\ref{fig:edge2}). We repeat this process until all the segments of the edge band are covered.

\begin{figure}[htbp]
    \centering
    \includegraphics[width=1\linewidth]{images/terre4.png}
    \caption{The green lines at original image identify border zones and angles to guide the patch selection in the patch database construction process.}
    \label{fig:line1}
\end{figure}

\begin{figure}[htbp]
        \centering
        \includegraphics[width=1\linewidth]{images/terre5.jpg}
        \caption{The Construction of new boundaries in the generated seismic images uses texture synthesis based on  selected samples from the seismic image patches database.}
        \label{fig:edge2}
    \end{figure}    

To synthesize the salt zones, we use the corresponding type of image region as a seismic texture reference to generate the new synthesized pixels. Finally, we synthesize the conventional rock zones using the corresponding sediment regions in the seismic texture sample provided as an example. We divide the seismic image synthesis into steps to conform to the specific characteristics of this kind of seismic image, such as the edge, saline rock, and conventional rock zones, as Fig.~\ref{fig:source1} shows.

\begin{figure}[htbp]
    \centering
    \includegraphics[width=1\linewidth]{images/terre6.png}
    \caption{Real seismic saline images divided by context zones used as input to the texture synthesis process.}
    \label{fig:source1}
\end{figure}


\subsection{Complete Seismic Image Synthesis Pipeline}

The complete synthesis pipeline integrates the VAE-based context generation and the context-oriented texture synthesis methods. The proposed seismic image synthesis algorithm processes all image zones, one by one, until the seismic image synthesis process is completed, according to their corresponding zone samples in the input seismic image data. The pipeline can be summarized as follows: (1) the VAE method generates a new salt body mask defining the geometric structure of the synthesized seismic image, (2) edge detection is used to identify boundary zones between salt and sediment, (3) the texture synthesis algorithm fills each zone (edges, salt, and sediment) with the appropriate seismic textures based on the training seismic image database, and finally (4) the final synthetic seismic image is assembled by combining all synthesized zones. This systematic approach, detailed in Algorithm~\ref{alg:synthesis}, which was designed to ensure that the synthesized seismic images maintain both structural consistency and textural realism.

\begin{algorithm}[htbp]
\caption{Context-Oriented Seismic Image Synthesis}
\label{alg:synthesis}
\begin{algorithmic}[1]
\State \textbf{Input:} Trained VAE model $D_\theta$, Texture Database $T_{db}$
\State \textbf{Output:} Synthetic Seismic Image $I_{synth}$

\vspace{0.2cm}
\Statex \textit{// Stage 1: Context Generation}
\State Sample latent vector $z \sim \mathcal{N}(0, I)$
\State Generate salt mask $M_{salt} \leftarrow D_\theta(z)$
\State Identify boundary mask $M_{boundary}$ from $M_{salt}$ via edge detection and dilation
\State Define sediment mask $M_{sediment} \leftarrow \neg(M_{salt} \cup M_{boundary})$
\State Initialize $I_{synth}$ as an empty image

\vspace{0.2cm}
\Statex \textit{// Stage 2: Context-Oriented Texture Synthesis}
\State \textit{// Synthesize Boundary Zone}
\For{each segment $s$ in $M_{boundary}$}
    \State Compute local orientation $\alpha_s$
    \State Select best patch $P_{boundary} \in T_{db}$ based on $\alpha_s$
    \State Synthesize texture for $s$ in $I_{synth}$ using $P_{boundary}$
\EndFor

\State \textit{// Synthesize Salt Zone}
\State Select salt patches $P_{salt} \in T_{db}$
\State Synthesize texture for region $M_{salt}$ in $I_{synth}$ using $P_{salt}$

\State \textit{// Synthesize Sediment Zone}
\State Select sediment patches $P_{sediment} \in T_{db}$
\State Synthesize texture for region $M_{sediment}$ in $I_{synth}$ using $P_{sediment}$

\vspace{0.2cm}
\State \Return $I_{synth}$
\end{algorithmic}
\end{algorithm}



\section{Results and Discussion}
\label{sec:results}

\PARstart{T}{his} section presents and discusses the obtained experimental results in order to illustrate the effectiveness of the proposed approach. The experimental evaluation encompasses a qualitative assessment performed by domain experts, and a quantitative comparison with the available state-of-the-art. This dual evaluation strategy follows the  criteria below :

\begin{enumerate}
    \item \textbf{Qualitative Evaluation by Experts}: Experts try to identify the portions of salt dome structures inside the seismic samples, and each expert produces a mask label that will be compared with the mask generated by our proposed method. Quantitative measures, such as the F1 score, are used to evaluate the difference between the experts predictions and the proposed method synthesized masks. 
    
    \item \textbf{Quantitative Comparison with the State-of-the-Art}:
    The quantitative comparison is performed with the method proposed by Ferreira \emph{et al.}~\cite{Ferreira2020} since it is the representative of the state of the art that addresses the problem being tackled in our work. This comparison  is based on three measures: mean squared error (MSE), structural similarity index (SSIM), and the Euclidean distance, using Local Binary Patterns (LBP) for texture attribute assessment. These measures quantify the similarity between the original seismic images and the sketch-guided synthetic samples generated by their GAN-based network~\cite{Ferreira2020}. We will evaluate our proposed method using the same measures to facilitate a direct comparison with this representative~\cite{Ferreira2020} of the state-of-the-art.   
\end{enumerate}


\subsection{Dataset}

The salt body dataset from the TGS Salt Identification Challenge is used in these experiments. It comprises 2D image slices representing a 3D view of the Earth's interior obtained through reflection seismology. The data is a set of images chosen randomly in the subsurface. The photos are 101 x 101 pixels and each pixel is classified as either salt or sediment \cite{TGS2023}.

\begin{figure}[htbp]
    \centering
    \includegraphics[width=1\linewidth]{images/terre7.png}
    \caption{Comparison between a real seismic image sample, and the geoscience specialist selection in terms of saline area and ground truth (saline mask).}
    \label{fig:expert1}
\end{figure}


\subsection{Evaluation Measures} \label{eval:meas:sect}

Through qualitative analysis, the geoscience specialist evaluates the seismic image samples to identify the portions of salt bodies in these images. To assess the quality of the seismic image synthesis, the precision, recall, and F1-score measures are used to compare the synthetic seismic image with the ground truth, as detailed below:

The F1 score is a widely used measure for evaluating binary classification models. It is particularly useful when a balance between precision and recall is desired. It allows access to the quality of the classification predictions. While the correct classification rate is simple to understand, it can be misleading in imbalanced datasets. Therefore, measures such as precision, recall, and the F1 score tend to provide a clearer scenario~\cite{Powers2011}.

\begin{itemize}
    \item \textbf{Precision:} is the proportion of true positives (TP) over all predicted positive results (TP + FP)
    \begin{equation}
    \text{Precision} = \frac{TP}{TP + FP}
    \end{equation}

    \item \textbf{Recall:} or sensitivity, is the proportion of true positives over all actual positive cases (TP + FN)
    \begin{equation}
    \text{Recall} = \frac{TP}{TP + FN}
    \end{equation}

    \item \textbf{F1 Score:} is the harmonic mean of precision and recall, providing a single measure that considers both. 

\end{itemize}
    \begin{equation}
    \text{F1 Score} = 2 \times \frac{\text{Precision} \times \text{Recall}}{\text{Precision} + \text{Recall}}
    \end{equation}
    
The F1 score is particularly valuable in classification situations, where the positive class tends to be rare. It helps to understand the trade-off between false positives and false negatives. Usually, the F1 score provides a balanced measure that is less susceptible to class distribution bias.

We use the precision, recall, and F1-score measures to evaluate the salt bodies masks generated by the experts, and we can compute them using the confusion matrix derived from these expert-generated masks. To evaluate the quality of the synthesized samples, we evaluate the overlap between the experts selections of pixels corresponding to the salt bodies in synthetic images and synthetic images masks, as illustrated in Fig.~\ref{fig:expert1}. Fig.~\ref{fig:expert2} illustrates the ground truth image and the specialist evaluation of the salt bodies locations, and the computed precision, recall, and F1-score measures for this particular case.

Regarding quantitative comparison, following the methodology proposed by Ferreira \emph{et al.}~\cite{Ferreira2020}, is used to assess the quality of the synthesized seismic image samples by comparing each synthetic image with the original seismic image that serves as a template. The template is the basis for producing the sketches, as shown in Fig.~\ref{fig:sketches}, which guide the seismic samples syntheses. To this end, we use the measures mean squared error (MSE), Structural Similarity Index (SSIM), and Euclidean distance between the original template image and the synthetic images based on the Local Binary Pattern (LBP) texture attribute.

\begin{figure}[htbp]
	\centering
	\includegraphics[width=1\linewidth]{images/terre8}
	\caption{Image sketches based on the original seismic templates.}
	\label{fig:sketches}
\end{figure}

The MSE defines a pixel-to-pixel distance between two images. We calculate it as follows:  subtract each synthetic pixel value from the real one, square each error (this eliminates negative signs and penalizes larger errors), and finally take the average of these squares:

\begin{equation}
\label{eq:mse}
\text{MSE} = \frac{1}{n} \sum_{i=1}^{n} (y_i - \hat{y}_i)^2,
\end{equation}
where $n$ is the total number of pixels, $y_i$ is the $i$-th pixel value in the original image, and $\hat{y}_i$ is the corresponding pixel value in the synthetic image.

The SSIM measures the perceptual similarity between two images $x$ and $y$ by incorporating texture information, such as variances and covariances, which is widely used in computer vision and image processing to assess the quality of image reconstructions, compressions, or transmissions. It evaluates luminance (brightness), contrast, and structural attributes as local patterns and shapes. Since SSIM values are defined in the range $[-1, 1]$, the common transformation below  represents it as a distance to facilitate comparison, and from now on, it is called DSSIM:

\begin{equation}
\label{eq:mse2}
\text{DSSIM}(x, y) = \frac{1 - \text{SSIM}(x, y)}{2}
\end{equation}

The interpretation of DSSIM values as a distance between 0 and 1 is as follows: 

\begin{itemize}
    \item DSSIM = 0: the images are identical;
    \item DSSIM from 0.005 to 0.10: high similarity;
    \item DSSIM > 0.25: low perceptual similarity;
    \item DSSIM from 0.5 to 1: very different.
\end{itemize}

The LBP is a more robust texture attribute successfully applied in medical image analysis and in geosciences \cite{Vatamanu2013,BrittoMattos2017}. The LBP calculation uses four neighbors, distance 1, and a 64-pixel tile size. The calculation produces a histogram for each image, representing the frequency of local binary patterns. Finally, we compute the Euclidean Distance between the histograms. Therefore, the smaller the distance, the more similar the textures of the two images are. Otherwise, larger histogram distances represent different patterns.

\begin{figure*}[htbp]
    \centering
    \includegraphics[width=1\textwidth]{images/terre9.png}
    \caption{Synthetic Seismic Samples.}
    \label{fig:placeholder}
\end{figure*}


\begin{figure}[htbp]
    \centering
    \includegraphics[width=1\linewidth]{images/terre10.png}
    \caption{Quantitative analysis illustration using sample measurements for comparying the ground truth and expert analysis: Precision: 0.88 ; Recall: 0.87 ; F1-score: 0.87.}
    \label{fig:expert2}
\end{figure}

\begin{table*}[htbp]
\centering
\caption{Result table: real images evaluation by the experts.}
\label{tab:real_images_evaluation}
\sisetup{output-decimal-marker = {,}}
\begin{tabular*}{\textwidth}{@{\extracolsep{\fill}}l S[table-format=1.5] S[table-format=1.5] S[table-format=1.5] S[table-format=1.5] S[table-format=1.5] S[table-format=1.5] S[table-format=1.5] S[table-format=1.5] S[table-format=1.5] S[table-format=1.5]@{}}
\toprule
& \multicolumn{3}{c}{\textbf{Expert 1}} & \multicolumn{3}{c}{\textbf{Expert 2}} & \multicolumn{3}{c}{\textbf{Expert 3}} & \textbf{Mean} \\
\cmidrule(lr){2-4} \cmidrule(lr){5-7} \cmidrule(lr){8-10}
\textbf{Measures} & {\textbf{Precision}} & {\textbf{Recall}} & {\textbf{F1-score}} & {\textbf{Precision}} & {\textbf{Recall}} & {\textbf{F1-score}} & {\textbf{Precision}} & {\textbf{Recall}} & {\textbf{F1-score}} & {\textbf{F1-score}} \\
\midrule
Std Dev & 0.06034 & 0.04584 & 0.04037 & 0.05429 & 0.04763 & 0.04408 & 0.04780 & 0.05288 & 0.03414 & 0.03953 \\
Mean & 0.91480 & 0.91563 & 0.91396 & 0.91845 & 0.90420 & 0.91064 & 0.82958 & 0.81400 & 0.82017 & 0.88159 \\
\bottomrule
\end{tabular*}
\end{table*}

\begin{table*}[htbp]
\centering
\caption{Result table: Synthetic images evaluation by the experts.}
\label{tab:synthetic_images_evaluation}
\sisetup{output-decimal-marker = {,}}
\begin{tabular*}{\textwidth}{@{\extracolsep{\fill}}l S[table-format=1.5] S[table-format=1.5] S[table-format=1.5] S[table-format=1.5] S[table-format=1.5] S[table-format=1.5] S[table-format=1.5] S[table-format=1.5] S[table-format=1.5] S[table-format=1.5]@{}}
\toprule
& \multicolumn{3}{c}{\textbf{Expert 1}} & \multicolumn{3}{c}{\textbf{Expert 2}} & \multicolumn{3}{c}{\textbf{Expert 3}} & \textbf{Mean} \\
\cmidrule(lr){2-4} \cmidrule(lr){5-7} \cmidrule(lr){8-10}
\textbf{Measures} & {\textbf{Precision}} & {\textbf{Recall}} & {\textbf{F1-score}} & {\textbf{Precision}} & {\textbf{Recall}} & {\textbf{F1-score}} & {\textbf{Precision}} & {\textbf{Recall}} & {\textbf{F1-score}} & {\textbf{F1-score}} \\
\midrule
Std Dev & 0.06706 & 0.04698 & 0.04509 & 0.05597 & 0.05397 & 0.04433 & 0.05221 & 0.05644 & 0.03889 & 0.04277 \\
Mean & 0.90439 & 0.89575 & 0.89868 & 0.88823 & 0.87772 & 0.88186 & 0.83354 & 0.82261 & 0.82648 & 0.86901 \\
\bottomrule
\end{tabular*}
\end{table*}


\subsection{Qualitative Evaluation}

We conducted the qualitative evaluation of the synthesized seismic image samples through a systematic process with the support of geoscientists, who are experts in the field. Our goal is to verify the synthesized samples quality, relevance, and reliability using the knowledge and experience of qualified professionals. Three geoscientists who routinely work in seismic analysis and interpretation participated in this evaluation. We conducted the evaluation by having experts reviewing the seismic images, and selecting the different regions types by painting the salt bodies using a graphics computer program. We then compare the area painted by the expert with the ground truth established for the corresponding saline body of the seismic image. Then, we assessed the specialists accuracies by having them identifying salt bodies on twenty original seismic images of the dataset. Afterwards, they selected and painted (marked) the same types of areas on a set of thirty random synthetic seismic images. In addition, we warned the experts that we add some images for control without any saline body.

To evaluate the results, the measures described in Section~\ref{eval:meas:sect} were calculated for each expert and for the group of experts, and we consolidate their averages and standard deviations. Table~\ref{tab:real_images_evaluation} shows these evaluations for 20 real seismic images, showing an average precision score of 0.88761, suggesting that experts identifications of saline rock portions were 88.7\% accurate, leaving 11.3\% misidentified as salt, which was actually common rock. The obtained average recall of 0.87795 indicates that 87.7\% of the saline rock portions were effectively identified, with a small remainder of 12.3\% missed salt portions. Considering the F1-score calculation, the specialists achieved an average score of 0.88159 for the real seismic images.

Table~\ref{tab:synthetic_images_evaluation} shows the evaluation  results for 30 synthetic seismic images, showing an average precision of 0.87539, which suggests that experts identifications of the synthetic saline rock portions were 87.5\% correct, with 12.5\% misidentified as salt, which actually was common rock. The average recall was 0.86536 in the synthetic seismic images, indicating that 86.5\% of the synthetic saline rock portions were effectively identified, and 13.5\% were missed since these pixels were not identified as salt. Finally, the average F1-score achieved 0.86901 for the synthetic images, considering all evaluations on synthetic images made by the experts.

Comparing the results of the geoscience specialist evaluation of real and synthetic seismic images, a small difference of less than 2\% can be observed in favor of the real images. Therefore, from the perspective of the specialists evaluations, the synthetic seismic images are virtually indistinguishable from the real seismic images. In that sense, we can conclude that, from the point of view of the geoscience specialists consulted, the synthetic images are comparable to the real seismic images.

\subsection{Quantitative Evaluation} 

Ferreira et al.~\cite{Ferreira2020} propose to design the synthesis experiments according to specific sketches types, or configurations of seismic regional types (see Fig.~\ref{fig:sketches}). Therefore, in order to make a direct and meaningful comparison between the proposed approach and the comparable state-of-the-art approach proposed by Ferreira et al.~\cite{Ferreira2020} (since both methods approach basically the same problem), we organized our experimental analysis into four groups of samples, each corresponding to a specific sketch type or configuration, as shown in Fig.~\ref{fig:sketches}, since these sketches are analogous to sketches evaluated in Ferreira et al.~\cite{Ferreira2020}. As proposed in Ferreira et al., different sketch types (A--E) should be used to guide the generation of synthetic seismic images, with each type representing a distinct level of structural and textural detail. Similarly, the four groups of samples used in our work reflect variations in input conditions or synthesis parameters that parallel the diversity of sketches in the baseline study~\cite{Ferreira2020}. This design of experiments in groups allows to systematically assess the robustness and adaptability of our proposed method under scenarios comparable to those available in the literature, and to provide a fair, group-wise performance comparison across key quantitative measures. The comparison results for each group are presented in detail in Fig.~\ref{fig:placeholder} and in Table~\ref{tab:metricsSummary}.

Considering the summary statistics presented in Table~\ref{tab:metricsSummary} for MSE, DSSIM, and LBP-DISTANCE across the four groups of seismic image regions and the totals, some conclusions may be obtained about the distribution of these measures. For the sake of clarity, Table~\ref{tab:metricsSummary} summarizes the values of these measures showing the minimum value ($min$), the largest value ($max$), the $1^{st.}$ and $3^{rd.}$ quantiles values ($Q1$ and $Q3$), as well as the median values ($median$), as welll as the totals considering all groups. The difference between $Q1$ and $Q3$ are named  Interquartile Range (IQR), and the difference between $max$ and $min$ are named simply as overall range.  

By analyzing the Mean Squared Error, group 3 achieved the best performance, consistently showing the lowest MSE, with a median of 542.87, indicating that this group 3 generally had the smallest errors. Group 4 has the highest median MSE at 1093.89, suggesting that it had the most significant typical errors and the worst performance observed. Group 3 is the most consistent, having the smallest Interquartile Range (IQR = 20.96) and the smallest overall range (109.04), meaning its errors are tightly clustered. Group 1 is the least consistent with a large overall range of 380.69. The total distribution shows larges IQR (362.28) and overall range (690.63), respectively, which is expected when combining the wide range of measures values from the individual groups.

Regarding DSSIM, which is a measure where lower values indicate greater similarity, Group 2 achieves the lowest median DSSIM of 0.2424, indicating the highest structural quality (least distortion/most similarity) compared to the other groups. Group 2 also is the most consistent, since it has the smallest IQR (0.0080) and overall range (0.0293). It is worth observing that all groups show relatively low variability in terms of DSSIM as compared to MSE.

Regarding LBP (Local Binary Pattern) Distance, a measure often used for texture analysis where a lower distance indicates higher similarity in texture features, Group 3 has the lowest median LBP Distance of 0.0800, indicating the highest fidelity in texture features. Group 3 is also the most consistent with the smallest IQR (0.0200), in a tie with Group 4. However, Group 3 has the smallest overall range (0.0700), making it the overall most stable in terms of texture-feature quality.

Based on the three measures, Group 3 consistently shows the best overall performance with the lowest median and lowest variability (IQR and overall range) for both MSE (error magnitude) and LBP-DISTANCE (texture quality). Group 2 is highly competitive, showing the best performance in terms of DSSIM, which indicates higher structural similarity. The box plot in Fig.~\ref{fig:boxplots} confirms the groups performances in the DSSIM measure.

%TODO

%LUCIANO: precisas mostrar o que estes grupos representam e como são formados  - Ferreira propos 5 grupos e nós 4 precisas explicar porque da diferença

\subsection{Comparison with the State-of-the-Art}

To provide a quantitative comparison, we evaluated our proposed method against the sketch-based approach of Ferreira \emph{et al.}~\cite{Ferreira2020}, our results are shown in Table~\ref{tab:metricsSummary} and the results obtained by Ferreira \emph{et al.}~\cite{Ferreira2020} are shown in Table I (obtained from their original work). Both methodologies measure the similarity between synthetic and original seismic images using three key measures: Mean Squared Error (MSE), Structural Similarity Dissimilarity (DSSIM), and Euclidean Distance based on Local Binary Patterns (LBP Distance). For these measures, lower values signify better performance.

The baseline method proposed by Ferreira \emph{et al.}~\cite{Ferreira2020} employs a Generative Adversarial Network (GAN) architecture to synthesize seismic images from user-provided sketches. Their approach consists of two main stages:

\paragraph{Sketch Generation:} The method begins with hand-drawn, or computer-generated, sketches that represent the desired geological structures in the target seismic image. These sketches serve as conditional inputs that guide the image generation process. The authors evaluated five different sketch types (A through E), varying in complexity and level of detail:

\begin{itemize}
    \item \textbf{Type A}: Basic contour sketches showing only the main boundaries;
    \item \textbf{Type B}: Enhanced contours with additional structural details;
    \item \textbf{Type C}: Sketches incorporating intensity gradients;
    \item \textbf{Type D}: Detailed sketches with textural hints;
    \item \textbf{Type E}: Comprehensive sketches combining structural and textural information.
\end{itemize}

\paragraph{GAN-Based Image Synthesis:} The core of the method proposed by Ferreira \emph{et al.}~\cite{Ferreira2020} employs a conditional GAN (cGAN) architecture based on the pix2pix framework. The generator network takes the sketch as input, and learns to produce realistic seismic images that match the sketches structural layout. The discriminator network evaluates whether the generated images are realistic and properly aligned with the input sketches. The training process involves:
\begin{enumerate}
    \item Training the generator to map sketches to realistic seismic images;
    \item Training the discriminator to distinguish between real and synthetic seismic images;
    \item Optimizing both networks through adversarial training until convergence.
\end{enumerate}

Their proposed GAN-based approach offers flexibility in controlling the output through user-defined sketches, but requires manual intervention to create appropriate sketch inputs. In contrast, our proposed VAE-based method with context-oriented texture synthesis operates autonomously, generating both the structural masks and the corresponding seismic textures without requiring manual sketch creation.


\subsubsection{Comparative Analysis}

In Ferreira \emph{et al.}~\cite{Ferreira2020} five sketch types (A--E) were analysed, and the sketches types D and E were found to produce the best results. In our work, we analyzed four distinct groups, with Group 3 achieving the better, most consistent, performance. The subsequent subsections present a detailed comparison based on the median values of the measurements, which offer a robust measure of each method typical performance.

\paragraph{Mean Squared Error (MSE)}

The Mean Squared Error (MSE) measures the pixel-wise distance between two images, with lower values indicating a closer match in intensity. While the sketch-based method achieved a median MSE of 4712.1 with their best-performing sketch type, our context-oriented approach yielded a median MSE of 542.87. This result is approximately 8.7 times lower, demonstrating a substantial improvement in the pixel-wise correspondence between the synthetic and original images.

\paragraph{Structural Similarity Distance (DSSIM)}

The DSSIM measure, derived from SSIM as shown in Eq.~\ref{eq:mse2}, transforms a structural similarity measure into a dissimilarity measure. In this context, DSSIM values closer to 0 indicate higher structural similarity between the images.

In Ferreira \emph{et al.}~\cite{Ferreira2020}, the best median DSSIM achieved was 0.39 (with Sketch Type D). Our method surpassed this benchmark, with Group 2 achieving a median DSSIM of 0.2424. This result is significantly better, as a DSSIM value above 0.25 can be interpreted as "low perceptual similarity." While the baseline result falls into this range, our Group 2 median is just below this threshold. Furthermore, Group 2 was not only the top performer in structural quality but also the most consistent, showing the smallest Interquartile Range (IQR) of 0.0080.

\paragraph{LBP Distance}

LBP Distance measures texture similarity by calculating the Euclidean distance between Local Binary Pattern (LBP) feature vectors. Lower values signify a closer match in textural features.

The baseline work in Ferreira \emph{et al.}~\cite{Ferreira2020} reported a lowest median LBP Distance of 0.17 (Sketch Type D). In contrast, our proposed method best result was a LBP Distance median of 0.0800 (Group 3). This value is less than half of the baseline, indicating that our method generated textures with substantially higher fidelity to the original images. Group 3 also demonstrated the most consistent texture feature quality.


\paragraph{Discussion}

Overall, our proposed context-oriented synthesis approach tends to obtain a better quantitative performance across all three measures when compared to the sketch-based method proposed in Ferreira \emph{et al.}~\cite{Ferreira2020}. The improvement is most pronounced in terms of the MSE and LBP distance measures, where our proposed method achieves significantly lower median values, suggesting notable gains in terms of pixel-level accuracy and texture fidelity.

It is important to note that quantitative measures may not always align with qualitative human assessment. For instance, the baseline study observed that while their Sketch Type D had the best quantitative scores for DSSIM and LBP Distance, their Sketch Type E produced more visually realistic images. In our work, we supplemented our quantitative findings with a detailed qualitative analysis by domain experts. The experts evaluations concluded that the synthetic images were virtually indistinguishable from the real ones, a finding supported by an F1-score difference of less than 2\%.


\begin{figure}[htbp]
	\centering
	\includegraphics[width=1\linewidth]{images/terre11}
	\caption{Box plot of DSSIM values for sketches groups.}
	\label{fig:boxplots}
\end{figure}


\begin{table}[htbp]
    \centering
    \caption{Summary statistics for MSE, DSSIM, and LBP Distance across four groups and total.}
    \label{tab:metricsSummary}
    \begin{tabular*}{\linewidth}{@{\extracolsep{\fill}}lccccc}
        \textbf{Metrics}
        & \textbf{Group 1} & \textbf{Group 2} & \textbf{Group 3} & \textbf{Group 4} & \textbf{Total} \\
		\midrule
		\multicolumn{6}{c}{\textbf{MSE}} \\
		\midrule
		Min    & 718.12  & 651.21 & 511.83 & 987.74  & 511.83 \\
		Q1     & 855.92  & 686.99 & 535.13 & 1056.76 & 643.62 \\
		Median & 916.33  & 718.47 & 542.87 & 1093.89 & 779.19 \\
		Q3     & 963.64  & 740.92 & 556.09 & 1127.13 & 1005.90 \\
		Max    & 1098.81 & 794.37 & 620.87 & 1202.46 & 1202.46 \\
		\midrule
		\multicolumn{6}{c}{\textbf{DSSIM}} \\
		\midrule
		Min    & 0.2544 & 0.2279 & 0.2608 & 0.2683 & 0.2279 \\
		Q1     & 0.2741 & 0.2395 & 0.2701 & 0.2842 & 0.2567 \\
		Median & 0.2814 & 0.2424 & 0.2746 & 0.2900 & 0.2766 \\
		Q3     & 0.2876 & 0.2475 & 0.2798 & 0.2947 & 0.2870 \\
		Max    & 0.3057 & 0.2572 & 0.2963 & 0.3098 & 0.3098 \\
		\midrule
		\multicolumn{6}{c}{\textbf{LBP - DISTANCE}} \\
		\midrule
		Min    & 0.0800 & 0.1200 & 0.0400 & 0.1900 & 0.0400 \\
		Q1     & 0.1500 & 0.1300 & 0.0700 & 0.2200 & 0.1043 \\
		Median & 0.2000 & 0.1500 & 0.0800 & 0.2300 & 0.1640 \\
		Q3     & 0.2500 & 0.1700 & 0.0900 & 0.2400 & 0.2215 \\
		Max    & 0.3900 & 0.2000 & 0.1100 & 0.2700 & 0.3900 \\
		\bottomrule
	\end{tabular*}
\end{table}
\section{Ablation Study}
\label{sec:ablation}

\PARstart{T}{o} assess the contribution of each component of our proposed methodology, we conducted an ablation study, discussed below. We design these experiments to isolate the impact of the VAE-based context generation and the context-oriented texture synthesis in the final quality of the synthetic seismic images.

\subsection{Impact of Context-Oriented Synthesis}

A key innovation of our method is the division of the synthesis process into three distinct zones: salt, conventional rock, and their boundary. To evaluate the importance of this separation, we conducted an experiment where this context-oriented approach is removed.

\begin{itemize}
    \item \textbf{Experiment}: A single, non-parametric texture synthesis model would be used to generate the entire image, guided only by the binary salt/non-salt mask produced by the VAE. This model would not differentiate between the interior of the salt dome, the surrounding rock, and the crucial boundary region.
    \item \textbf{Hypothesis}: We expect a significant degradation in image quality. The model would likely struggle to reproduce the sharp, geologically complex interfaces between salt and sediment, resulting in blurred or unrealistic boundaries.
    \item \textbf{Evaluation}: The results would be compared against our full method using the same quantitative measures (MSE, DSSIM, LBP Distance) and qualitative expert analysis. A poorer performance would validate the necessity of our context-oriented approach.
\end{itemize}

\subsection{Contribution of the VAE for Mask Generation}

Our method relies on a VAE to generate realistic and diverse salt body geometries. The ablation study quantifies the benefit of this component as compared to simpler alternatives.

\begin{itemize}
    \item \textbf{Experiment}: Replace the VAE with a more basic mask generation technique. For instance, new masks could be created by applying simple geometric transformations (\emph{e.g.}, rotation, scaling, elastic deformation) to the existing ground-truth masks from the training set.
    \item \textbf{Hypothesis}: While simpler, this approach would likely yield a less diverse and geologically plausible set of masks. We hypothesize that the VAE ability to learn a rich latent distribution of salt geometries is crucial for generating high-quality and varied synthetic samples.
    \item \textbf{Evaluation}: The quality of the final synthetic images generated using these simpler masks would be assessed. A lower score in expert evaluations and quantitative measures would underscore the VAE contribution to the realism and diversity of the augmented dataset.
\end{itemize}

\subsection{Relevance of Boundary-Specific Synthesis}

The proposed method gives special treatment to the boundary zone, using patch selection oriented along local angles to construct the interface. The importance of this specific step can be evaluated as follows:

\begin{itemize}
    \item \textbf{Experiment}: We disable the angle-aware patch selection for the boundary zone, and synthesize the boundary using the same standard non-parametric texture synthesis applied to the salt and rock interiors, without considering the local orientation of the interface.
    \item \textbf{Hypothesis}: We predict that this change would lead to less coherent and realistic boundaries. We would lose the fine-grained textural details that align with the curve of the salt dome, resulting in visible artifacts and a less convincing transition between the geological layers.
    \item \textbf{Evaluation}: A direct comparison of the boundary regions between this ablated version and the full model, both qualitatively by experts and quantitatively (if a suitable measure for boundary coherence is defined), would highlight the value of this specialized synthesis step.
\end{itemize}

These ablation studies systematically validate our design choices and provide a deeper understanding of why the proposed context-oriented approach is effective for synthesizing high-fidelity seismic images.

% LUCIANO: ok propostos os ablation studies MAS TENS QUE MOSTRAR OS RESULTADOS !!!!

AQUIQ

\section{Concluding Remarks and Future Directions}
\label{sec:conclusion}

\PARstart{T}{his} work introduced a novel context-oriented data augmentation methodology for synthesizing labeled seismic images containing salt dome structures, addressing a critical challenge in offshore petroleum exploration where annotated training data is scarce and expensive to obtain. By integrating Variational Autoencoders for geometric mask generation with context-aware non-parametric texture synthesis, the proposed approach successfully generates synthetic seismic images that preserve both structural integrity and textural fidelity across three distinct geological zones: salt bodies, conventional rocks, and their boundaries.

The comprehensive evaluation framework, combining expert qualitative assessment with rigorous quantitative measures, validates the effectiveness of our methodology. Three geoscience experts evaluated the synthetic samples, achieving F1-scores that differed by less than 2\% from their evaluations of real images (0.86901 vs. 0.88159), demonstrating that the synthetic images are virtually indistinguishable from authentic seismic data. Quantitatively, our method substantially outperforms the state-of-the-art sketch-based GAN approach, achieving approximately 8.7 times lower MSE (542.87 vs. 4712.1), superior structural similarity with DSSIM of 0.2424 compared to 0.39, and more than 50\% improvement in texture fidelity with LBP Distance of 0.0800 versus 0.17.

The significance of this work extends beyond technical measures. In the petroleum industry, where seismic interpretation requires extensive expertise and labeled datasets are prohibitively expensive, our automated synthesis method offers a practical solution for generating high-quality training data at scale. The context-oriented approach ensures geological plausibility by respecting the distinct characteristics of different rock types and their interfaces, a crucial requirement for real-world applications in subsurface exploration and reservoir characterization.

\subsection{Applications}

The proposed methodology enables several important applications in the geoscience domain. The synthetic samples can augment limited real datasets to improve the robustness and generalization of deep learning models for salt body segmentation, such as U-Net and ResNet architectures, particularly where obtaining additional labeled data is impractical. This synthetic data can also serve as a pre-training corpus for transfer learning, allowing models trained on one geological basin to adapt more effectively to different regions with limited labeled samples.

Furthermore, the VAE's ability to interpolate in latent space enables the generation of salt geometries that are underrepresented in training datasets, helping to address class imbalance and improve model performance on edge cases. Beyond model training, these high-quality synthetic seismic images can support educational programs for geoscience professionals, providing diverse examples for teaching seismic interpretation and salt body identification skills. Finally, by generating multiple plausible interpretations of ambiguous seismic regions, the method can support uncertainty analysis in exploration decision-making processes.


\subsection{Future Research Directions}

While this work demonstrates promising results on the TGS Salt Identification Challenge dataset, several avenues for future research merit exploration:

Adapting the methodology to synthesize other complex geological features such as faults, channels, and stratigraphic layers would broaden its applicability across different exploration contexts through \textbf{Extended Geological Structures}. Building upon this foundation, developing hierarchical \textbf{Multi-scale Synthesis} approaches that can generate seismic volumes at multiple resolutions could better capture the multi-scale nature of geological structures and improve computational efficiency.

To enhance the method's generalizability and practical utility, investigating \textbf{Domain Adaptation} techniques to adapt the synthesis process across different seismic acquisition parameters, geological basins, and imaging conditions represents a crucial direction. Furthermore, incorporating additional conditioning variables such as depth, geological age, or depositional environment through \textbf{Conditional Generation} could provide finer control over the synthesis process and generate more geologically consistent samples.

From a practical deployment perspective, conducting systematic studies on how synthetic data augmentation affects the performance of various seismic interpretation tasks through \textbf{Downstream Task Evaluation}—including not only segmentation but also classification, detection, and attribute prediction—would validate the method's broader impact. Additionally, optimizing the computational efficiency of the pipeline to enable \textbf{Real-time Synthesis} during active learning workflows, where models request specific types of synthetic samples to address their current weaknesses, would significantly enhance its practical value.

Looking forward, integrating seismic wave propagation physics and rock property constraints into the generation process through \textbf{Physics-informed Constraints} could further enhance the geological realism and interpretability of synthetic samples, representing a promising direction that bridges data-driven and physics-based approaches.

In conclusion, this work establishes a robust foundation for context-oriented seismic image synthesis, demonstrating that carefully designed generative approaches can produce training data of sufficient quality to support advanced machine learning applications in geoscience. The combination of automated mask generation, zone-specific texture synthesis, and rigorous evaluation provides a template for developing similar methodologies in other domains where labeled data scarcity constrains the application of deep learning techniques.


\begin{thebibliography}{00}

\bibitem{Zeng2019}
Y.~Zeng, K.~Jiang, and J.~Chen, ``Automatic seismic salt interpretation with deep convolutional neural networks,'' in \emph{Int. Conf. Inf. Syst. Data Min. - ICISDM 2019}, 2019, pp. 16--20, doi: 10.1145/3325917.3325926.

\bibitem{He2016}
K.~He, X.~Zhang, S.~Ren, and J.~Sun, ``Deep residual learning for image recognition,'' in \emph{2016 IEEE Conference on Computer Vision and Pattern Recognition (CVPR)}, 2016, pp. 770--778, doi: 10.1109/CVPR.2016.90.

\bibitem{Ronneberger2015}
O.~Ronneberger, P.~Fischer, and T.~Brox, ``U-Net: Convolutional networks for biomedical image segmentation,'' in \emph{Medical Image Computing and Computer-Assisted Intervention -- MICCAI 2015}, 2015, pp. 234--241.

\bibitem{Henriques2021}
L.~Henriques, S.~Colcher, R.~Milidiú, A.~Bulcão, and P.~Barros, ``Generating data augmentation samples for semantic segmentation of salt bodies in a synthetic seismic image dataset,'' arXiv, Jun. 2021. [Online]. Available: http://arxiv.org/abs/2106.08269

\bibitem{Li2020}
K.~Li, S.~Chen, and G.~Hu, ``Seismic labeled data expansion using variational autoencoders,'' \emph{Artif. Intell. Geosci.}, vol. 1, pp. 24--30, Dec. 2020, doi: 10.1016/j.aiig.2020.12.002.

\bibitem{Gatys2015}
L.~Gatys, A.~S.~Ecker, and M.~Bethge, ``Texture synthesis using convolutional neural networks,'' in \emph{Advances in Neural Information Processing Systems}, Curran Associates, Inc., 2015.

\bibitem{Zhou2018}
Y.~Zhou, Z.~Zhu, X.~Bai, D.~Lischinski, D.~Cohen-Or, and H.~Huang, ``Non-stationary texture synthesis by adversarial expansion,'' \emph{ACM Trans. Graph.}, vol. 37, no. 4, pp. 49:1--49:13, Jul. 2018, doi: 10.1145/3197517.3201285.

\bibitem{Powers2011}
D.~M.~W.~Powers, ``Evaluation: From precision, recall and F-measure to ROC, informedness, markedness \& correlation,'' \emph{Journal of Machine Learning Technologies}, 2011.

\bibitem{Kingma2014}
D.~P.~Kingma and M.~Welling, ``Auto-encoding variational Bayes,'' arXiv, May 2014, doi: 10.48550/arXiv.1312.6114.

\bibitem{Efros1999}
A.~A.~Efros and T.~K.~Leung, ``Texture synthesis by non-parametric sampling,'' in \emph{Proceedings of the Seventh IEEE International Conference on Computer Vision}, Kerkyra, Greece, 1999, vol. 2, pp. 1033--1038, doi: 10.1109/ICCV.1999.790383.

\bibitem{Wang2021}
T.~Wang, D.~Trugman, and Y.~Lin, ``SeismoGen: Seismic waveform synthesis using GAN with application to seismic data augmentation,'' \emph{Journal of Geophysical Research: Solid Earth}, vol. 126, no. 4, p. e2020JB020077, 2021, doi: 10.1029/2020JB020077.

\bibitem{Anjom2024}
F.~Khosro Anjom, F.~Vaccarino, and L.~V.~Socco, ``Machine learning for seismic exploration: Where are we and how far are we from the holy grail?'' \emph{GEOPHYSICS}, vol. 89, no. 1, pp. WA157--WA178, Jan. 2024, doi: 10.1190/geo2023-0129.1.

\bibitem{Choi2025}
B.~Choi, S.~Pyun, W.~Choi, and Y.~Cho, ``Synthetic seismic data generation with pix2pix for enhanced fault detection model training,'' \emph{Computers \& Geosciences}, vol. 197, p. 105879, Mar. 2025, doi: 10.1016/j.cageo.2025.105879.

\bibitem{Zhou2024}
Y.~Zhou, R.~Xiao, D.~Lischinski, D.~Cohen-Or, and H.~Huang, ``Generating non-stationary textures using self-rectification,'' in \emph{Proceedings of the IEEE/CVF Conference on Computer Vision and Pattern Recognition}, 2024, pp. 7767--7776.

\bibitem{TGS2023}
TGS Kaggle, ``TGS Salt Identification Challenge | Kaggle,'' 2023. [Online]. Available: https://www.kaggle.com/competitions/tgs-salt-identification-challenge/data

\bibitem{Ferreira2020}
R.~S.~Ferreira, J.~Noce, D.~A.~B.~Oliveira, and E.~V.~Brazil, ``Generating sketch-based synthetic seismic images with generative adversarial networks,'' \emph{IEEE Geoscience and Remote Sensing Letters}, vol. 17, no. 8, pp. 1460--1464, Aug. 2020, doi: 10.1109/LGRS.2019.2945680.

\bibitem{Vatamanu2013}
O.~A.~Vatamanu, M.~Frandes, M.~Ionescu, and S.~Apostol, ``Content-based image retrieval using local binary pattern, intensity histogram and color coherence vector,'' in \emph{2013 E-Health and Bioengineering Conference (EHB)}, Nov. 2013, pp. 1--6, doi: 10.1109/EHB.2013.6707396.

\bibitem{BrittoMattos2017}
A.~Britto Mattos, R.~S.~Ferreira, R.~M.~Da Gama e Silva, M.~Riva, and E.~V.~Brazil, ``Assessing texture descriptors for seismic image retrieval,'' in \emph{2017 30th SIBGRAPI Conference on Graphics, Patterns and Images (SIBGRAPI)}, Oct. 2017, pp. 292--299, doi: 10.1109/SIBGRAPI.2017.45.

\end{thebibliography}

\begin{IEEEbiography}[{\includegraphics[width=1in,height=1.25in, clip,keepaspectratio]{images/terre.jpg}}]{Luciano D. Terres}
Received the B.S. degree in Computer Science from the Federal University of Rio
Grande do Sul (UFRGS), Brazil, in 1996. In 2010 received the M.S. degree in
Engineering and Computer Systems from the Federal University of Rio de Janeiro (UFRJ-COPPE), Brazil.
Currently, he is a Ph.D. student in Computer Science at UFRGS. Since 2005 has been a researcher at Petrobras Cenpes Research Center.
His research interests include petroleum exploration, petroleum systems simulation, computer vision, image processing, and pattern recognition.

\end{IEEEbiography}

\begin{IEEEbiography}[{\includegraphics
[width=1in,height=1.25in,clip,
keepaspectratio]{images/schar.jpg}}]
{Jacob Scharcanski}
received the B.Sc. degree in electrical engineering in 1981 and the M.Sc.
degree in computer science in 1987, both from the Federal University of Rio
Grande do Sul (UFRGS), Brazil, and the Ph.D. degree in systems design
engineering from the University of Waterloo, Canada, in 1993. Currently, he
is a Full Professor in computer science at UFRGS.  He has authored and
co-authored over 170 refereed journal and conference papers, and has
contributed to several books on imaging and measurements. In addition to his
academic publications, he has several technology transfers to the private
sector. Presently, he serves as an Associate Editor for two journals and has
served on dozens of International Conference Committees. Prof. Scharcanski is
a Senior Member of the IEEE, and served as an IEEE IMS Distinguished Lecturer
on several occasions. His areas of expertise are Image Processing, Pattern
Recognition, Imaging Measurements and Computational Methods in Finance.
\end{IEEEbiography}

\EOD
\end{document}
